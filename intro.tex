%!TEX root = thesis.tex

\chapter{Introduction}

\epigraph{In my experience I found that the most effective way to express
something in order to make others understand is to use the simplest language.
Also I learned from teaching that the more rigid the language the less
effective it is.}{\textcite{mahathera1990}}

\section{Introduction to the central themes of the thesis}

Before launching into detail into the various topics under investigation, this
section briefly touches on the issues addressed over the entire thesis:
vibrations and their isolation and suppression in
\secref{vibrations-summary-intro}; permanent magnets and their role in the
design of supporting structures and other devices in
\secref{magnets-summary-intro}; and `\qzs' systems, which unify in this thesis
the fields of vibrations and magnets, in \secref{qzs-summary-intro}. These
three broad areas are then investigated in more detail in \secref{literature-explore}.

\subsection{The problem of vibrations}
\seclabel{vibrations-summary-intro}

The disturbing effects of vibrations are a well-known and continuing problem.
The transmission of vibrations from a source can only ever be reduced, not eliminated — that is, not without removing the source entirely from the local region affected. And in many cases there is no single source; the very ground itself may be a medium through which undesirable vibrations are transmitted.
Earthquakes are an extreme example of this, but on a smaller scale there are continuous time-varying displacements of the `fixed' ground beneath us.
\note{Not to be too earth-centric, but many of the ideas here have a different relevance in off-planet circumstances.}
The earth should in fact be considered as a distributed vibrating structure, of very great mass in total, that has a range of displacement profiles dependent on the local surrounding impedance conditions. 

Whether the source of the disturbance is near or far, or how it propagates
through the ground to arrive at the region of interest, these disturbances can
cause a variety in problems on equipment that is required to, ideally, remain
absolutely still. A good example is during the electro-lithography performed
to construct our microchips, in which nanometre-sized disturbances can affect
the overall yield of the silicon wafer produced. Mitigating the effects of a
disturbance through the base on which a structure is supported is known as
`vibration isolation' and is the over-arching problem in which the work of
this thesis should be put in context.

A contrasting vibration problem occurs when some manner of machine causes its own vibration; a well-known example is a washing machine that exerts an oscillating disturbance force on itself through a mass imbalance. This type of vibration problem requires a rather different set of design solutions and often its solution acts in opposition to the vibration isolation problem discussed above. Reducing the effects of self-induced vibration disturbance will be termed `vibration suppression'
\note{The descriptor `Vibration isolation' is commonly used to refer to both problems, but it would be confusing here to avoid the clarification.}
for the purposes of this thesis and will be revisited on occassion herein.

There are a variety of `classic' solutions for both vibration isolation and
vibration suppression. A particularly simple solution for \emph{both} problems is to mount the equipment on a many-ton slab of concrete. This is not always practical. Another common approach for vibration isolation is to support the equipment with pneumatic springs. When in operation, these springs provide a low supporting stiffness and low static deflection; they can typically be used to support hundreds of kilograms with resonance frequency of less than five hertz.

Other support methods besides pneumatics are able to achieve low stiffness; this is an area that will be further investigated in the literature review. But one method in particular is interesting in this context: permanent magnets can provide low stiffness support without energy expenditure. Their nonlinear forces in both attraction and repulsion allows the possibility of interesting supporting designs, and their non-contact nature allows their use in vacuum and `clean-room' environments.

\subsection{Permanent magnets used for mechanical design}
\seclabel{magnets-summary-intro}

The last twenty years has seen the maturation of the rare earth permanent magnet industry. These magnets are now widely available in large sizes and strong magnetisation at relatively low prices. They are now used in a variety of mechanical design, including bearings, couplings, \maglev\ trains, and so on, which take advantage of non-contact attractive and/or repulsive forces.
Magnets can be used to make other magnets move or to keep them in place — there is little limit to the ingenuity of their application. However, in part to this complex behaviour between them, there are few design guidelines that can be used to aid their use [quote Moskowitz].

We can speak broadly about their integrated use in force design: magnets can be used in conjunction with current-carrying coils to effect time-varying forces (as in shakers and speakers); soft iron can be used to guide the magnetic fields into desired regions or away from unwanted areas (\eg, latches and motors); or magnets can be used alone for unique force--displacement characteristics or simply for applying non-contact forces (\eg, bearings).

These ideas in mechanics and dynamics have application back to the field of vibrations. A `synergy' between the two fields is seen in areas such as energy harvesting from ambient vibration, the study of vibration in high-speed magnetic bearings, and one of the main themes of this thesis — nonlinear and/or noncontact forces for support equipment for vibration isolation.

Support mass with a noncontact force can also be called `levitation', an area which deserves its own mention. \textcite{earnshaw1842} proved that levitation with the force of permanent magnets alone was impossible, although this did not become common knowledge
\note{If it can even be said to be `commonly known' today. Anecdotal evidence suggests otherwise.}
until much later — given by the range of patents issued that assume the opposite. Exceptions to `Earshaw's Theorem' include the use of diamagnetic materials and actively-controlled system, amongst some others. It is the possibility of overcoming instability with active means that is of interest in this thesis.

\subsection{\QZS\ systems}
\seclabel{qzs-summary-intro}

The transition between stable and unstable forces becomes interesting in the context of vibration isolation. Between positive and negative stiffness in a force--displacement characteristic, there is a inflexion point of zero stiffness. This point is termed a `\qzs' position to emphasise that the dynamic behaviour of the system in this condition can be rather complex and usually unstable. `True' zero stiffness would imply \emph{no} connection between between the mass and the base, as if they were floating in free space — the motion of one would have no effect on the motion of the other.

As systems approach \qzs, their vibration isolation inproves as the resonant frequency decreases. Operation at the \qzs\ position is not possible as the system is, at best, only marginally stable, and the system must be tuned (based on the applied loading) to achieve best results.

Certain magnetic systems are not the only ones to exhibit \qzs. The phenomenon was first proposed using inclined springs to achieve a `buckling' effect. Magnets are more convenient in many ways than inclined or buckling springs in that the negative stiffness can be applied directly without having to exploit the byproduct of a mechanical spring or linkage arrangement, which can be more bulky.

Active control systems can be used with \qzs\ systems to improve their performance in one of three ways:
\begin{enumerate}
  \item Standard active vibration control with velocity feedback;
  \item Remove the instability at the \qzs\ location with a nonlinear controller;
  \item Online tuning of the system for load-independent operation.
\end{enumerate}
The first two of these strategies are investigated in this thesis.

\subsection{Project context}

The original goal of this project was to design and build a vibration
isolation table using non-contact magnetic springs. This goal can be
split into two: the design of a non-contact magnetic spring (suitable
for a vibration isolation table); and the design of the vibration
isolation table itself.

Vibration isolation tables are generally designed to attenuate natural
disturbances from the ground to the tabletop. Current commercial models use
pneumatic springs to perform this task, and this project arose out of
curiosity: could magnetic springs be used instead?

Using magnets for load bearing brings its own set of challenges. For
completely non-contact support, active control must be used to
stabilise \emph{at least} one degree of freedom. For the design to be
worth investigating, some advantage to using magnets should also be
demonstrated.\footnote{Although I took much pleasure in explaining
  over the years that my \PhD\ project was to `build a table that
  floats on magnets'.}

However, the field of active control has been well-established and the feat of building a stabilising controller for a system with relatively simple dynamics is not worthy of the research for a \PhD. The work presented in this thesis is the investigations around the idea of building a `table that floats on magnets' while pulling out enough interesting nuggets to prove worthy of the title of `research'.

\section{Exploring the literature in the context of the ideas discussed}
\seclabel{literature-explore}

\subsection{V}

\subsection{Levitation}

Before looking at the general class of magnet force devices, levitation is a good place to start.

\subsubsection{Magnetic levitation is impossible!}
\seclabel{earnshaw}

The act of passively levitating a magnet by another is well
known as impossible, although popular unlearned opinion is not
aware of the fact.  \textcite{earnshaw1842} proved that
objects in the influence of fields that apply forces with an
inverse-square relation to displacement cannot form
configurations of stable levitation. Approximately one hundred
years later, \textcite{tonks1940} wrote a paper reminding his
contemporaries of the work of \citeauthor{earnshaw1842} by
applying the proof specifically to the field of magnetics:
\begin{quote}
\dots no flexible assemblage of magnetic poles, in which
readjustments in position of the poles in the group can
occur, can be stable in either a fixed field or in the field
from another such assemblage\dots
\end{quote}
The proof is conceptually quite simple. We start with the
equation for the magnetic field; when there are no external
current terms, it can be shown to be expressed as Laplace's
equation:
\begin{dmath}[compact]
\grad\magB = 0 \implies \divgrad\magB = 0
\end{dmath}.  
The potential energy of a magnet is proportional
to the magnetic field it is subjected to, $U =
-\magM\bdot\magB$, so when the magnetisation is time-invariant 
(as in the case of a permanent magnet), we have:
\begin{dmath}[compact,label=earnshaw]
\divgrad U = \divgradxyz{U} = 0 
\end{dmath}.  
The double differentiations of the energy are
the stiffnesses in each direction.  But for stable
equilibrium, these three terms must be greater than zero. This
cannot satisfy \eqref{earnshaw} and thus levitation cannot
occur.

\begin{figure}
  \grf[width=0.4\textwidth]{Figures/Theory/saddle}
  \caption{A ball in unstable equilibrium on a saddle-shaped curve.}
  \figlabel{saddle}
\end{figure}

This is easy to visualise by analogy. \figref{saddle} shows a ball balancing
on a saddle-shaped curve, which we can take to be potential energy of a \twoD\
system. Clearly it is stable in one direction, unstable in the other.
Perturbations `left' or `right' will result in reaction forces keeping it
centred and stable, whereas small displacements `into' or `out from' the page
will result in increased perturbation to instability.

\subsubsection{Exceptions to Earnshaw}

Earnshaw's proof does not rule out all forms of `levitation' unconditionally,
however. \textcite{boerdijk1956a} reviewed the known methods for levitation,
covering levitation by gravitation forces, pressure reaction forces, radiation
field forces, and finally in detail, various magnetic and electromagnetic
forces.

Because Earnshaw's theorem looks only at the case for static equilibrium,
cases when the magnetic field is dynamic are not covered. This can occur
broadly under two circumstances: when the magnetic field is generated with
\AC\ currents; and when an unstable permanent magnet arrangement is stabilised
with an active control system.

Levitation using a magnetic field produced by \AC\ currents was covered in
detail by \textcite{laithwaite1965}. The basic mechanism of this type of
levitation is that \AC\ currents create dynamic magnetic fields that induce
eddy currents in a levitating object, and it is the interaction of the
magnetic fields of these induced currents that causes the levitation. The
technique uses a large amount of power, and is not especially suitable for the
purposes of this research for this reason.

For an actively stabilised levitation system, the levitation forces are
created by permanent magnets (which have time-constant magnetic fields) and
the necessary stabilisation applied with variable current electromagnets with
a feedback control system. For this reason the magnetic fields are known as
quasi-static. This system was first implemented by Holmes who levitated a
magnetic needle, as cited by \textcite{boerdijk1956a}. Some more practical
examples of these types of system are covered in the next section.

This summary is fairly brief; \textcite{bleuler1992} wrote a more detailed
overview. His paper introduces the `self-sensing active magnetic bearing'
\cite{vischer1993} which uses back--electromotive force from the controlling
electromagnet to sense the position of the floating element. This eliminates
the need for a more classical position sensor (\eg, optical or capacitive),
but the control system is necessarily more complex and the behaviour not as
precise.

Some Italian guy showed that you can levitate a ring magnet above another by
using vibrations to find small zone of stability in the nonlinear dynamics
\cite{bassani2007}.

\subsubsection{Diamagnetic forces}

Levitations involving diamagnetic material are also exempt from Earnshaw's
theorem. This was the motivation for the papers of
\textcite{boerdijk1956b,boerdijk1956a} in which he cites Braunbek, who derived
that magnetic material is governed by Earnshaw's theorem only because it has a
relative magnetic permeability ($\permMag$) greater than one—\ie, a
permeability greater than that of the surrounding medium. Material with
$\permMag<1$ is \emph{not} covered by the theorem since the magnetic flux from
the diamagnetic material becomes dependent on the displacement of the
permanent magnet; this violates the condition of Earnshaw's theorem that fixed
magnetic fields be used, and so static levitation involving magnets and such
diamagnetic material is very possible. To demonstrate this,
\citeauthor{boerdijk1956b} levitated a small cylindrical magnet of dimensions
\diameter$\,\SI{1}{mm} \times \SI{0.3}{mm}$.

More contemporary studies on diamagnetic levitation look at the
levitation of larger objects including strawberries and frogs
\cite{berry1997, simon2000, simon2001}, using very high-powered
electromagnets (on the order of \SI{2}{T} \fixme{check}).

Unfortunately, these techniques are not suitable for large load
bearing.  Even the most diamagnetic substance known, pure bismuth, has
$\permMag \approx 0.9998$ — hardly different than that of air. The forces
exchanged via magnetic flux between magnetic and diamagnetic
materials, therefore, are incredibly small and not suited at all to
the purposes of this research.

Superconducting material, on the other hand, behaves ideally diamagnetic with
$\permMag =0$, so the forces produced between a superconductor and a magnet
are equal to the forces between two (equal) permanent magnets themselves. This
allows many exciting possibilities for stable levitation. However, such
materials must be cooled to very low temperatures (\emph{at least} below
\SI{-150}{\celsius}) in order to remain superconductive. Such a requirement
renders this method financially and functionally impractical for this
research. A review of work in this area has been published by
\textcite{ma2003}.

\subsection{Applications of magnetic forces}

\textcite{coey2002} discusses a broad range of magnetic applications,
summarised in \tabref{magnet-applications}.

\begin{table}
\begin{wide}
\begin{tabular}{@{}lll@{}}
\toprule
Field & Magnetic effect & Examples \\
\midrule
Uniform & Zeeman splitting & Magnetic resonance imaging \\
& Torque & Alignment of magnetic powder \\
& Hall effect, magnetoresistance & Sensors, read-heads \\
& Force on conductor & Dynamic Motors, actuators, loudspeakers \\
& Induced emf & Generators, microphones \\
Nonuniform & Force on charged particles & Beam control, 
radiation sources %(microwave, ultra-violet; X-ray) 
\\
& Force on magnet & Bearings, couplings, Maglev \\
& Force on paramagnet & Mineral separation \\
Time varying & Varying field & Dynamic Magnetometers \\
& Force on iron & Dynamic Switchable clamps, holding magnets \\
& Eddy currents & Metal separation, brakes \\
\bottomrule
\end{tabular}
\end{wide}
\caption{Applications of permanent magnet materials, 
adapted from \textcite{coey2002}.}
\tablabel{magnet-applications}
\end{table}

That magnets can apply forces to one another over a distance is quite a novel
concept in a mechanical world accustomed to friction. It has been a short
while, relatively speaking, that it has been possible to even \emph{produce}
magnets with enough coercive force to apply useful mechanical forces.
Non-contact magnetics in mechanical systems is advantageous due to high
precision and wear-free operation due to lack of friction. This section looks
broadly at some of the main applications of the field.

\subsubsection{Maglev transportation}

The largest body of research into magnetic levitation is on so-called `maglev'
transportation. Its well-known goal is to use a levitated train or car to
provide extremely fast and efficient transportation. This field, which is
rather diverse in terms of the techniques under investigation, is finally now
achieving commercial application in the real world after some 20 or 30 years
of research. While it has some concepts in common with this research (large
loads, magnets), the techniques used tend to be rather distanced from those
that will be applied for this project because they focus on transportation
rather than \emph{elimination} of movement.

\subsubsection{Magnetic actuators}

In more recent years, another application for magnetic levitation has been
investigated, which is the precision control of a levitated platform. Commonly
cited for use in the semiconductor industry for photolithography, these
levitators were first researched around 20 years ago.

The first designs allowed travel in a single direction, \eg,
\textcite{trumper1992}, while more recent developments allow more directions
of control. Such devices are capable of supporting small loads, and applying
horizontal translation forces to effect displacements of up to around
\SI{200}{mm} with nanometre precision. Two planar devices are invented in the
independent theses of \textcite{kim1997} and \textcite{molenaar2000}. Most
recently, a six degree of freedom non-contact actuator was demonstrated by
\textcite{verma2004}. See also the device built by \textcite{kim2007} to
support around \SI{2}{kg} over a travel of $\SI{5}{mm}\times\SI{5}{mm}$ with
nanometre precision. The high performance and mechanical simplicity of their
design is note-worthy.

Here're some more to look at: \textcite{boeij2008,zhang2008a}

The reason these devices are unsuitable for this research is due to their
travelling capability. Rather than using the primarly magnetic flux for load
support, these designs use it in order to provide positioning control in the
horizontal directions.

Here's something new: \textcite{shameli2008}.

A recent six degree of freedom actuator has been demonstrated
by \textcite{jansen2008}, which supports loads in the order of
\SI{10}{kg} with a relatively large air gap (\SI{1}{mm}--\SI{2}{mm}) 
and large stroke ($\SI{230}{mm}\times\SI{230}{mm}$).

[\textcite{dasilveira2005}] Analytical expression for the normal force
between two magnets on a back-iron plate and a perpendicular coil. The system
is for a planar actuator, and the motivation is to be able to determine the
amount of out-of-plane force generated by a particular design. Could very well
be useful for some of my ideas.

\subsubsection{Interesting devices}

I like magnetic levitation of objects in a wind tunnel
\cite{higuchi2008}.

Magnetic fields can be used for a vast array of scientific uses. For example,
noncontact sensing of material properties that involve variable conductivity,
including fatigue cracks, defects in printed circuit boards, and even plastic
landmine detection \cite{mukhopadhyay2005}.

Here would be a good place to cite the brain, nerve, stuff, etc
\parencite{lu2008,demachi2008}


Here's a fun application of magnetics where a computer input device is built
with magnetic sensors placed on the wrist in order to sense single finger-tip
motion from the opposite hand \parencite{han2008}.

I don't know if I care, but \textcite{vanwest2007} created a
haptic interface for manuipulating small objects with magnetic
levitation using the so-called `zero power' technique well
known due to Mizuno.

\textcite{park2008} demonstrates a MIMO controller for a flywheel energy
storage mechanism using magnetic bearings while applying active vibration
isolation.

\textcite{tomie2005} This paper's a bit of fun; uses a magnet attached
to a cantilever, excited by an external field in a water tank, to propel a
robotic fish. Only left-right oscillations were produced, so the fish was
constrained to move in a plane; two turning methods are investigated.


\subsection{Exploring \qzs\ systems}
\seclabel{vibrations-qzs}

Addition of negative stiffness elements in a design reduces the resonance
frequency, which improves vibration isolation. Early examples of such designs
using inclined springs were shown by \textcite{molyneux1957}. Such systems may
tuned to achieve a local region of zero stiffness, which is often termed
`\qzs'. \textcite{alabuzhev1989} looked at the nonlinear behaviour of such
systems, and also more recently by
\textcite{carrella2006,carrella2007,carrella2008} and improved by
\textcite{kovacic2008}. More recently, the dynamic response of these systems
has been analysed \parencite{carrella2009,carrella2008thesis} and shown to
have prominent nonlinearities that distort the frequency response but that do
not decrease the vibration isolation efficacy in general.

The use of buckling beams as a negative stiffness element to achieve \qzs\ has
also been implemented in practice~\cite{platus1999,tarnai2003,lee2007}.

A brief discussion of zero stiffness elements in hand-held vibrating
machinery yields perhaps little insight into our problem
\cite{sokolov2007}, but mentions that friction is the biggest problem
once zero stiffness has actually been achieved; this problem is
obviated when non-contact supports are used.

An interesting collection of literature for very low frequency vibration
isolation resulted from various gravity wave interferometers built around the
world. Some of these used the negative stiffness of the buckling beam, in
various forms, to reduce the resonance frequency of an isolator
\cite{cella2005}.

\QZS\ can also be achieved with magnetic systems. Magnetic configurations with
negative stiffness can be used to augment a positive stiffness support (which
can be simply a conventional spring) to lower the resonance frequency. For
example, \textcite{beccaria1997} used this technique (under the term `magnetic
antisprings') to improve the isolation for gravity wave detectors.
\textcite{carrella2007a,carrella2008} has also used a attractive magnets in
parallel with conventional springs to reduce the resonance frequency of the
system. Purely non-contact magnetic systems can also be used to similar effect
\cite{robertson2006,robertson2007}. Generally, systems that use the negative
stiffness between attracting magnets cannot be brought to a stable \qzs\
region due to their `softening spring' characteristic.

\textcite{hol2006} discuss a `gravity compensator' that uses the idea of an
axial bearing with \ang{90} rotated magnetisations to bear load in the
vertical direction. (Also see \textcite{yonnet1981} for related bearings that
I hadn't thought about in enough detail yet.) This creates something like a
negative quadratic force relationship, which has zero stiffness in the centred
vertical position.

Analysis of the nonlinear dynamics of such systems such as performed by
\textcite{lee2004b,kovacic2008} can be quite involved and is outside the scope
of this research.


\section{Digression}

The method for literature review taken in this thesis may be considered rather
eccentric. During the course of my research, I couldn't help but notice the
flood of literature to be an almost unstoppable flow of information that was
rather difficult to remain `on top of'. Certainly required some expenditure of
effort on my part. And perhaps to my detriment, the bibliography in this
thesis is rather large — although certainly not exceeding that of a single
literature review in certain specific fields.

This thesis is a drop of water in the ocean of this literature that is
increasing every month. It will be a defining problem in the next few decades,
I believe, to attempt a unification and consolidation of the amount of
information that is currently available and, also, and more importantly, the
way that new work is published.

\section{Notation}

The mathematics in this thesis is typeset consistently but somewhat
differently than might be expected. Here's an arbitrary example:
\begin{dmath*}
f\fn{x} = \half\gp{x+\Sin{x}}.
\end{dmath*}


Here's a nice big glossary section that will take me fucking ages to get
right. Maybe I should just scrap it.

\section{Typesetting}

The typesetting of this thesis has been carefully designed with the following main objectives:
\begin{enumerate}
\item Ease of reading content, and
\item Ease of finding information.
\end{enumerate}
For the first point, a highly legible font has been chosen and the size of the
text calculated for optimal reading properties. For the second point, the
outer margins of the page have been utilised for holding `markers' such as
equation numbers and section numbers; this places such information close the
edge of the paper to aid `flicking through' the thesis. It also moves them
away from the main text where their encroachment of the material in the main
text is less useful.

In the \PDF\ version of this document, clickable hyperlinks have been inserted
in all relevant cross-referencing situations to aid navigation.

\section{Paraphrasing David Foster Wallace}

David Foster Wallace discusses the feeling of writing a book.
\note{\url{http://www.badgerinternet.com/~bobkat/jesty.html}}
His words work equally well for a thesis:

\begin{quotation}
You love [it] very much. And you want others to love it, too, when the time
finally comes for [it] to go out and face the world. But wanting other people
to love it, now, means hoping that others somehow won't see [it] as you see it
\dash as a grotesque, malformed betrayal of the very possibilities that
spawned it. You want to sort of fool people: you want them to see as perfect
what you in your heart know is a betrayal of all perfection. Or else you don't
want to fool these people; [...] You want to be terribly wrong: you want [its]
hideousness to turn out to have been nothing but your own weird delusion or
hallucination. But that'd mean you were crazy ... But worse: it'd also mean
you see and despise hideousness in a thing you made (and love), in your spawn,
in certain ways you. But that's still what you most want: to be completely,
insanely, suicidally wrong.

But it's still all a lot of fun. Don't get me wrong.
\end{quotation}
