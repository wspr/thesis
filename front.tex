%!TEX root = thesis.tex

\thispagestyle{empty}
\phantomsection
\label{titlepage}
\bookmark[dest=titlepage]{Title page}
\calccentering{\unitlength}
\begin{adjustwidth*}{\unitlength}{-\unitlength}
\setlength{\parindent}{0pt}

\begin{flushright}
  \grf[width=50mm]{Figures/Logo/ualogo_mono.pdf}
\end{flushright}

\vfill

Faculty of Engineering,
Computer and Mathematical Sciences\\
\textsc{school of mechanical engineering}

\vfill

{\Large\raggedright\csname @title\endcsname}

\vspace{10mm}

Will Robertson\\
Ph.D.\ Thesis
\vspace{10mm}

\today

\vfill

\vfill

\begin{tabular}{@{}ll}
Supervisors:    & Assoc.\,Prof.\ Ben Cazzolato  \\
                & Assoc.\,Prof.\ Anthony Zander
\end{tabular}
\end{adjustwidth*}

\newpage
\thispagestyle{empty}
\null
\vfill
\begin{quote}
  \LARGE
  \makebox[0pt][r]{`}\textit{Table that floats on magnets}'

  \vfill
  \normalsize
  \raggedright
  Copyright \textcopyright\ 2003--2013 Will Robertson
  and The University of Adelaide

  \bigskip
  Document generated \today.

  \bigskip
  School of Mechanical Engineering\\
  University of Adelaide SA\\
  Australia 5005\\
\end{quote}

\vfill
\begin{quotation}
  \noindent This document has been prepared and typeset using \LaTeX{}.

  Page margins have been chosen as a trade-off between achieving the
  optimal number of characters per line for ease of reading, and trying
  to fit the typeblock onto the poorly-sized (for books), although
  convenient, \acro{A4} stock.
  \note{For reference, see \textit{The Elements of Typographic
      Style} by Robert Bringhurst.}
  pdf\/\TeX's margin kerning is used to ensure optical straightness of these
  margins.

  If you are reading this electronically, \PDF\ hyperlinks have
  been automatically inserted in all appropriate cross-referencing positions.
\end{quotation}
\vfill
