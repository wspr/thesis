%!TEX root = thesis.tex

\chapter{References}

To add coherency to the bibliography and to aid browsing, references are
ordered alphabetically by first author and grouped by major topic:
\begin{description} 
  \item[-- Vibrations] on page \secpageref{refvib}, 
  \item[-- Levitation] on page \secpageref{reflev}, 
  \item[-- Forces] on page \secpageref{refforce}, and 
  \item[-- Other] (none of the above) on page \secpageref{refother}. 
\end{description}

Some citations fall into more than one category (this very thesis is an
example \cite{robertson2009-thesis}), in which case the reference will appear in
multiple locations in the bibliography; these are conspicuous as their
citation numbers will be out-of-order.

\defbibheading{levitation}{%
  \section{Levitation; magnetic and otherwise}%
  \seclabel{reflev}}
\defbibheading{forces}{%
  \section{Forces between magnets and electromagnets}%
  \seclabel{refforce}}
\defbibheading{vibrations}{%
  \section{Active and passive vibration}%
  \seclabel{refvib}}
\defbibheading{other}{\section{Other}\seclabel{refother}}

% this allows the use of double upright quotes inside bib titles:
\DeleteQuotes
\MakeInnerQuote{"}

\printbibliography[
  maxnames=9,minnames=9,
  heading=vibrations,
  keyword=Vibrations]
  
\printbibliography[
  maxnames=9,minnames=9,
  heading=levitation,
  keyword=Levitation]
  
\printbibliography[
  maxnames=9,minnames=9,
  heading=forces,
  keyword=Forces]
  
\printbibliography[
  maxnames=9,minnames=9,
  heading=other,
  notkeyword=Vibrations,notkeyword=Levitation,notkeyword=Forces]

