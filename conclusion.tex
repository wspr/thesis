%!TEX root = thesis.tex

\addtocontents{toc}{%
  \protect \ifshorttoc
  \protect   \null
  \protect   \vspace{3ex}%
  \protect \fi
}
\part*{Fin}
%\bookmarksetup{startatroot}


\chapter{Conclusion}

\epigraph{
  I can't grasp much of anything without putting down my thoughts in writing, so I had to actually get my hands working and write these words.
}{\textcite{murakami2008}}

\chapterprecis{And future work.}

This thesis covers several areas of investigation in

\section{Future work}

This work leaves open many areas of investigation for work to be done in the future.
Some of these are highlighted here.

\begin{enumerate}
\item
Consolidating theory for the forces between magnets for additional magnet shapes.
The first task here would be to provide a well-tested and efficient implementation to calculate the axial and radial forces between cylindrical magnets given axial and radial displacement.
To the best of my knowledge, there are no results on the forces between spherical magnets.

\item
Investigate the non-ideal effects seen in real life with respect to magnet forces; \eg, what is the influence between magnet size and homogeneous magnetisation?
What are the practical limitations to designing multipole arrays — will demagnetisation occur in some cases, and if so to what extent?

\item
Are there closed form solutions to the eddy current problem for permanent magnets in the influence of finite thickness coils?
After taking inductance effects into account, how does this change the optimality for solutions found for electromagnetic coil design?

\item
The experimental results shown for a \qzs/ magnetic system are for a single \dof/ only; a project to extend these results to six \dof/ is already under way.

\item
Load balancing for a \qzs/ is required for practical application.
This is being investigated separately.

\item
Is it possible to design a nonlinear controller to maintain \qzs/ for a magnetic device?
Intuitively, there may never be a stable system that achieves perfect \qzs/; therefore, what are the limits to which such a system can be bound?

\end{enumerate}

